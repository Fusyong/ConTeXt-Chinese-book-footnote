%%%% 全局配置
\startenvironment % 与第一行代码间必须有空行 BUG

% %%%%%%%%%%%%%%%%%%%%%%%%%%%%%%%%%%%%%%%%%%%%%%
% 模式
% %%%%%%%%%%%%%%%%%%%%%%%%%%%%%%%%%%%%%%%%%%%%%%
% 校对模式,yes no keep(保持预先设置,其后调整无效,可在命令行设置)
\definemode[proofread][keep]
\enablemode[proofread]


% %%%%%%%%%%%%%%%%%%%%%%%%%%%%%%%%%%%%%%%%%%%%%%
% 设置变量 TODO
% %%%%%%%%%%%%%%%%%%%%%%%%%%%%%%%%%%%%%%%%%%%%%%
\setvariables[settings][%
    % 如果紧跟在startenvironment后,上行末尾需要注释,
    % 且注释前不能有空格,否则会引入空白页
    smallblank=quarterline,% 板块间的最小间距
    bodyfont=10.5pt,%正文字号 A5a类21—16,b类14,c类12,d类10.5
    width_em=36, %一行字数
    interlinespace=1.94em,
]
\def\smallblank{\blank[\getvariable{settings}{smallblank}]}
% 此处samepage在某些地方不起效,而在当地手动加则有效 TODO
\def\smallblanksamepage{\blank[\getvariable{settings}{smallblank},samepage]}

% %%%%%%%%%%%%%%%%%%%%%%%%%%%%%%%%%%%%%%%%%%%%%%
% 纸张、版面、页码、书眉
% %%%%%%%%%%%%%%%%%%%%%%%%%%%%%%%%%%%%%%%%%%%%%%
\definepapersize[mypaper][width=184 mm,height=260 mm]%成品184*260,出血36mm
\setuppapersize[mypaper][mypaper,oversized] % A4版面,A4纸+oversized(1.5cm)打印(扩展以便看到裁切线)
\definemeasure[Bleed][3mm]%出血
\definemeasure[Trim][7.5mm]%剪裁线偏置等于 oversized
\setuplayout[%
    backspace=30mm,% 内侧空白
    % 版心width,排中文时要取正文字号的整数倍,即`每行字数*字号`
    % A4正常可用width约160mm,约合14pt*33字*0.35146mm/pt=162.4mm
    % A5a类106*168,b类108*167,c类106*164,d类107*166
    % width={\dimexpr\getvariable{settings}{bodyfont}*\getvariable{settings}{width_em}\relax},
    % 或直取M宽/方空铅的整数倍,即`每行字数em`(与上法有极小的差别,与字体有关)
    % A5a类19*20,b类22*22,c类25*25,d类29*28
    width=\getvariable{settings}{width_em}em,
    leftmargin=0em,
    rightmargin=0em,%4em
    topspace=15mm,
    height=230mm,%A4 250
        header=7.5mm,
        footer=7.5mm,%7.5mm
    marking=on, % 裁切线,打印纸设置大于版面时才能显示
    location=middle, % 居中放置在纸张上
    bleedoffset=\measure{Bleed},
    trimoffset=-\measure{Trim},
]
% 页码
% !!!alternative=doublesided指定对页交替
% (一般setup­layout设置默认指奇数页,对页交替使偶数页采用奇数页的镜像值)
\setuppagenumbering[alternative=doublesided,location={footer,right}, state=start]
\setupuserpagenumber[alternative=doublesided,location={footer,right}]
% 文前用小写罗马字
\definestructureconversionset [frontpart:pagenumber][][r]
\definestructureconversionset [bodypart:pagenumber][][n]
\startsectionblockenvironment [bodypart]
    \setcounter[userpage][1]%重排正文页码
\stopsectionblockenvironment

% 书眉
\setupheadertexts
    []%单码右侧{- \hskip0em \userpagenumber \hskip0.25em -}
    [{\ttx\getmarking[jie]}]%单码左侧
    [{\ttx\getmarking[zhang]}]%双码右侧
    []%双码左侧{- \hskip0em \userpagenumber \hskip0.25em -}

% %%%%%%%%%%%%%%%%%%%%%%%%%%%%%%%%%%%%%%%%%%%%%%
% 中文配置
% %%%%%%%%%%%%%%%%%%%%%%%%%%%%%%%%%%%%%%%%%%%%%%
\mainlanguage[cn]
\language[cn]
\setscript[hanzi] % 汉字处理脚本(断行等)
\setupalign[hanging,hz] %行末标点悬挂

% 边沿margin/edge文本
\setupmargindata[style={\hwx\setupinterlinespace[line=1em]\setscript[hanzi]}]

% %%%%%%%%%%%%%%%%%%%%%%%%%%%%%%%%%%%%%%%%%%%%%%
% 字体和符号配置
% %%%%%%%%%%%%%%%%%%%%%%%%%%%%%%%%%%%%%%%%%%%%%%
% 字体配置
\environment[notoplus.tex]
\setupbodyfont   [notoplus,rm,\getvariable{settings}{bodyfont}]
% \environment[notoplus.tex]
% \setupbodyfont   [notoplus,rm,\getvariable{settings}{bodyfont}]
% \usetypescriptfile[mscore]
% \setupbodyfont   [mschinese-light,\getvariable{settings}{bodyfont}]

% 注释设置,支持中文断行
\startsetups[footnote:hanzi]
    \setscript[hanzi]
    \setupinterlinespace[line=1.5em]
\stopsetups

\def\bracketednumber#1{[#1]}
\def\bracketednumberup#1{\high{\bracketednumber{#1}}}
% 配置注释环境(note environment)的格式,包括正文中的注释标记和注释文本所在的容器
\setupnote[footnote][%
    setups={footnote:hanzi},
    textcommand=\bracketednumberup,
    ]
%  配置注释内容(note insert)的格式,即注释文本本身的样式
\setupnotation[footnote][%
    % way=bypage,%按页计数
    numbercommand=\bracketednumberup,%或通过left right
]

% %%%%%%%%%%%%%%%%%%%%%%%%%%%%%%%%%%%%%%%%%%%%%%
% 唯一名称(用于自动生成不重复的宏、引用等)
% %%%%%%%%%%%%%%%%%%%%%%%%%%%%%%%%%%%%%%%%%%%%%%
\definecounter[uniquecounter][way=bytext]
% \setnumber[uniquecounter][1]
\def\incrementuniquename{\incrementcounter[uniquecounter]}
\def\uniquename{uniquename\Characters{\rawcountervalue[uniquecounter]}}%


\definedataset[footnotetable][delay=yes] % 最后实现,会插入页码信息
%     
% }
\def\basefootnote[#1][#2]{%
    \incrementuniquename%
    \setdataset[footnotetable][buf={#1},page={#2},uniquename={\uniquename}]
    \ctxlua{userdata.basefootnote('#1', '#2', '\uniquename')}%
    }
\ReadFile{footnote.lua}


% 第一版
% % 同某页注[索引][页码] TODO 判断同页、后页
% \def\sameasfootnote[#1][#2]{%
%     % {\footnote{同\at[#1]页[\in{}{}[#1]]#2.}}%
%     {\high{同\at[#1]页[\in{}{}[#1]]#2.}}%
%     }
% 同某页注[索引][页码] TODO 判断同页、后页

\def\sameasfootnote[#1][#2]{\basefootnote[#1][#2]}


% 定义画勾画叉符号及字体
\definefont[checkMarkFont][name:dejavusansmono*default]
\define\TrueMark{{\checkMarkFont ✔}}
\define\FalseMark{{\checkMarkFont ✗}}
% 着重号 TODO 居中
\definebar[zhuozhong][text=\lower0.4em\hbox{\hskip0.45em .}, repeat=no]

% %%%%%%%%%%%%%%%%%%%%%%%%%%%%%%%%%%%%%%%%%%%%%%
% 设置颜色
% %%%%%%%%%%%%%%%%%%%%%%%%%%%%%%%%%%%%%%%%%%%%%%
\setupcolors[state=start] % 开启颜色(默认关闭,转成灰度)
% !!!如果不设置透明度,用于下画线时,字色透明度会覆盖下画线的透明度
\definecolor[secondColor]     [c=1,t=1,a=1] % 第二色(青色)
\definecolor[lightSecondColor][c=0.4,t=0.8,a=1] % 浅二色(青色)
\definecolor[aboveColor]      [c=1,t=0.8,a=1] % 上层色(80%透明的青色)
\definecolor[lightAboveColor] [c=0.4,t=0.8,a=1] % 上层色(80%透明的青色)
\definecolor[proofColor]      [m=1,t=0.8,a=1] % 校对文本色(80%透明的品色)
\definecolor[lightProofColor] [m=0.4,t=0.8,a=1] % 校对文本色(80%透明的品色)
\definecolor[transparentColor][k=0,t=0,a=1] % 透明色(就地隐藏文本用)

% %%%%%%%%%%%%%%%%%%%%%%%%%%%%%%%%%%%%%%%%%%%%%%
% 结构编号转换集合(各级标题编号的呈现方式)
% %%%%%%%%%%%%%%%%%%%%%%%%%%%%%%%%%%%%%%%%%%%%%%
% \definecounter[subsubsectioncounter][way=bysection] % 每section重起
% \defineconversion[cnbysection][\convertnumber{cn}{\rawcountervalue[subsubsectioncounter]}]
\definestructureconversionset[cnconversion]
    [R,  % part                部分,大写罗马数字
    cn,  % chapter             章/单元 中文数字
    ,    % section             节/课文标题
    ,    % subsection          小节/分组标题 
    cn,  % subsubsection       大题 中文数字
    n,   % subsubsubsection    小题 印度数码
    n,]   % subsubsubsubsection 小小题 印度数码
    [r] % Default setting 小写罗马字

% %%%%%%%%%%%%%%%%%%%%%%%%%%%%%%%%%%%%%%%%%%%%%%
% 设置标题
% %%%%%%%%%%%%%%%%%%%%%%%%%%%%%%%%%%%%%%%%%%%%%%
% 统一设置(个别标题后面可单独重设)
% \defineresetset[myresetset][1,1,1,1,0][1] %关闭第3级标题的序号重置
% \defineresetset[subsubsectionresetset][][0] %关闭,用于单个标题的设置
\setupheads
    [part,chapter,section,subsection,subsubsection,subsubsubsection]
    [%
    % after=, % 覆盖其后插入的空
    sectionconversionset=cnconversion,%不要混入空格
    number=no, % 不显示序号
    indentnext=yes, % 其后首行缩进
    sectionsegments=current, % 序号段:只包括当前序号
    % sectionresetset=myresetset, % 序号重置集合
    ]
% 部分重设
\setupheads
    [subsection,subsubsection,subsubsubsection]
    [%
    % before=, % 覆盖其前插入的空
    ]
% 单元设置,书眉留空,新页起 , page=yes新页/left偶数页/right奇数页(默认)
\definepagebreak [rightblankpage] [yes,header,footer,right]%定义新建页面的规则,再加一个footer会压制白页前有正文页的页脚
\setuphead[chapter][header=empty,page=rightblankpage,style={\setupalign[middle]\tfb}]
% 课文标题设置
\setuphead[section][style={\setupalign[middle]\tfa}]
% 分组/栏目标题设置
\setuphead[subsection][style={\bf}, number=no,distance=0pt]
% 大题设置
\setuplabeltext[cn][subsubsection={{},{、}}] % 前后标签
%after={\blank[samepage]}无效,受到dati后续命令的干扰
\setuphead[subsubsection][style={\hw}, number=no,distance=0pt,]


% %%%%%%%%%
% 自定义标题
% %%%%%%%%%
% 章
\definehead[zhang][chapter]%[command={\externalfigure[dummy][width=1cm,height=1cm]}]%插图
% 目录中的章注释(用在文本流中,章标题后)
\def\muluzhangzhu#1{%
    \writebetweenlist[zhang][location=here]{%
    \vskip0.5em%不能用 \blank[0.5*line]
    {\startnarrower[left]%
    \setupinterlinespace[line=1.4em]%
    \tt #1%\noindent 
    \stopnarrower}}}%

% 节
\definehead[jie][section]

% 小节
\definehead[xiaojie][subsection]%subsection subsubject
    [commandbefore={\hskip2em},before={\smallblank}, after={\smallblank}]

% 小节
\definehead[xiaoxiaojie][subsubsection]%subsection subsubject
    [commandbefore={\hskip2em},before={\smallblank}, after={\smallblank}]


% 自订文章标题(6级)、作者(7级)
\definehead[centersection][subsubsubsubject][align=middle, before=, after=,]
\definehead[authorsection][subsubsubsubsubject][align=middle, before=, after=,]
% \definehead[leftsection][subsubsubject][align=flushleft]
\def\essayhead[#1][#2]{\centersection{#1}%
    \doifnotempty{#2}{\authorsection{#2}}%
    }
% \define[2]\rhymehead{\leftsection{#1}\leftaligned{\itx #2}\blank[0cm]}
% \define[2]\classicalpoemhead{\centersection{#1}\middlealigned{\itx #2}\blank[0cm]}
% 短文标题
\def\centralTitle[#1][#2]{\subsubsubject{\tf }}

% % 设置列表组itemize、所有级别、阿拉伯数字、堆叠、其后不留空
% \setupitemgroup[itemize][each][n,packed][before={\smallblank}, after={\smallblank},]
% % 设置二级列表
% \setupitemgroup[itemize][2][left=(, right=),stopper=]
% % 定义答案列表,大写字母序号、横排、三列
% \defineitemgroup[answeritems]
% \setupitemgroup[answeritems][each][A,horizontal,three]

% %%%%%%%%%%%%%%%%%%%%%%%%%%%%%%%%%%%%%%%%%%%%%%
% 设置目录、PDF书签、交互、引用
% %%%%%%%%%%%%%%%%%%%%%%%%%%%%%%%%%%%%%%%%%%%%%%
\setupheadtext[content=目录]
\setupcombinedlist[content][list={zhang,jie},alternative=c]%
\setuplist[zhang][style=bold]
\setuplist[jie][margin=2em]
% pdf交互/链接
\setupinteraction[%
    state=start,
    color=,
    contrastcolor=,%同页引用的颜色,默认红色
    ]
% pdf书签
% \setupinteractionscreen[option={bookmark}]
\setupinteractionscreen[
    option={doublesided,bookmark},
    width=max,height=max,% necessary for Trim/BleedBox
]
\placebookmarks[zhang,jie,xiaojie,xiaoxiaojie,chapter,section,subsection,subsubsection] % 可加入自定义\definelist[mylist]
% 设置引用
% \setupreferencing[left=,right=] % 覆盖左右两侧的引号
% \define[1]\see{\at{}{页}[#1]\about[#1]}% 另见

% %%%%%%%%%%%%%%%%%%%%%%%%%%%%%%%%%%%%%%%%%%%%%%
% 空白、间距和缩进
% %%%%%%%%%%%%%%%%%%%%%%%%%%%%%%%%%%%%%%%%%%%%%%
% 段落间距
\setupwhitespace[none]
% 条目间距(覆盖预设)
% \setupitemgroup[packed]
% 行距设置 A5a类19*20,b类22*22,c类25*25,d类29*28
\setupinterlinespace[line=\getvariable{settings}{interlinespace}]
% 缩进设置
\setupindenting[yes, 2em, first]
% 窄行、缩进设置(每一级的缩进量)
\setupnarrower[left=01em]%
\setupcolumns[n=2, separator=rule]

% 引文
\setupdelimitedtext[blockquote][style={\tt},rightmargin=0em]

% 落款
\definedelimitedtext
  [luokuan]
  [
      leftmargin=\dimexpr(\getvariable{settings}{width_em}em*1/2),
      rightmargin=0em,  %
      style={\tt},
      left={},
      right={},
      before={\blank[medium]\startalignment[center]}, % 前面加空白
      after={\blank[medium]\stopalignment},% 后面加空白
   ]  


% 标准一字宽空格
\def\q{\hskip0.75em}

% 三星场景分割行
\def\threeasterisksline{%
    \testpage[3]%此处至少有三行,否则强制分页
    \midaligned{\stretched[width=4.5em]{* * *}}}

% % %%%%%%%%%%%%%%%%%%%%%%%%%%%%%%%%%%%%%%%%%%%%%%
% % 其他设置
% % %%%%%%%%%%%%%%%%%%%%%%%%%%%%%%%%%%%%%%%%%%%%%%
% % 下画线颜色
% \setupthinrules[color=secondColor]
% % 表格
% \definextable[mytable][offset=0.2em, option={stretch, width}, align={center,lohi}]

% %%%%%%%%%%%%%%%%%%%%%%%%%%%%%%%%%%%%%%%%%%%%%%
% 使用竖排、标点压缩、加注三模块(保持模块顺序)
% %%%%%%%%%%%%%%%%%%%%%%%%%%%%%%%%%%%%%%%%%%%%%%
% 竖排
% \usemodule[vtypeset]
% 
% 标点压缩与支持
\usemodule[zhpunc][pattern=quanjiao, spacequad=0.5, hangjian=false]
% 
% 四种标点压缩方案:全角、开明、半角、原样:
%   pattern: quanjiao(default), kaiming, banjiao, yuanyang
% 行间标点(转换`、,。.:!;?`到行间,pattern建议用banjiao):
%   hangjian: false(default), true
% 加空宽度(角):
%   spacequad: 0.5(default)
% 
% 行间书名号和专名号(\bar实例):
%   \zhuanmh{专名}
%   \shumh{书名}
% 
% 夹注
% \usemodule[jiazhu][fontname=tf, fontsize=10.5pt, interlinespace=0.2em]
% default: fontname=tf, fontsize=10.5pt, interlinespace=0.08em(行间标点时约0.2em)
% fontname和fontsize与\switchtobodyfont的对应参数一致
% 夹注命令:
%   \jiazh{夹注}
% %%%%%%%%%%%%%%%%%%%%%%%%%%%%%%%%%%%%%%%%%%%%%%

% % %%%%%%%%%%%%%%%%%%%%%%%%%%%%%%%%%%%%%%%%%%%%%%
% % 唯一名称(用于自动生成不重复的宏、引用等)
% % %%%%%%%%%%%%%%%%%%%%%%%%%%%%%%%%%%%%%%%%%%%%%%
% \definecounter[uniquecounter][way=bytext]
% % \setnumber[uniquecounter][1]
% \def\incrementuniquename{\incrementcounter[uniquecounter]}
% \def\uniquename{uniquename\Characters{\rawcountervalue[uniquecounter]}}%

% % %%%%%%%%%%%%%%%%%%%%%%%%%%%%%%%%%%%%%%%%%%%%%%
% % 存取答案数据
% % %%%%%%%%%%%%%%%%%%%%%%%%%%%%%%%%%%%%%%%%%%%%%%
% \definedataset[answertable][delay=yes] % 最后实现,会插入页码信息???
% \def\saveAnswer[#1,#2]#3{%
%     \doifnotmode{answersheet}{%
%         % \incrementuniquename%
%         % \expandafter\def\csname \strippedcsname\uniquename\endcsname{\someheadnumber[subsubsection][current]}%
%         \setdataset[answertable][cmd={#1}, ref={#2}, answer={#3}]%花括号可以避免数据中的符号引起错误
%         }%
% }

% 校对模式的存疑标记
\def\cunyi{\dosingleempty\docunyi}
\def\docunyi[#1]#2{%
    \dontleavehmode%
    \doifmodeelse{proofread}%
        {\color[proofColor]{%
            \offset[width=0em,y=-1em,location={right,top}]{%
                \hbox to 0pt{\rmx\iffirstargument #1 \fi ??}}%
            #2}}%
        {#2}%
}%


% %%%%%%%%%%%%%%%%%%%%%%%%%%%%%%%%%%%%%%%%%%%%%%
% 资源
% %%%%%%%%%%%%%%%%%%%%%%%%%%%%%%%%%%%%%%%%%%%%%%
% \useMPlibrary[dum]%占位图库
% \setupexternalfigures[location={local,default}]
% \useexternalfigure[sine][sin.pdf][width=3cm]
% \externalfigure[sine]

% 插图设置
\setupcaption
  [figure]
  [style=\ttx]
% \setuplabeltext[en][figure={图 }]

% 在汉字和非汉字间插入空格
% 把双符破折号加成三符
\startluacode
Thirddata = Thirddata or {}

local glyph_id = nodes.nodecodes.glyph
local node_insertbefore = node.insertbefore
local node_insertafter = node.insertafter
local node_new = node.new
local tex_sp = tex.sp

local function ischinesechar(c)
    -- cjk汉字,不含符号
    -- for more ranges:
    -- https://wiki.contextgarden.net/List_of_Unicode_blocks
    -- 按常用程度排列
    return (c >= 0x04E00 and c <= 0x09FFF) --CJK Unified Ideographs; cjkunifiedideographs
        or (c >= 0x03400 and c <= 0x04DBF) --CJK Unified Ideographs Extension A; cjkunifiedideographsextensiona
        or (c >= 0x02E80 and c <= 0x02EFF) --CJK Radicals Supplement; cjkradicalssupplement
        or (c >= 0x031C0 and c <= 0x031EF) --CJK Strokes; cjkstrokes
        or (c >= 0x0F900 and c <= 0x0FAFF) --CJK Compatibility Ideographs; cjkcompatibilityideographs
        or (c >= 0x20000 and c <= 0x2A6DF) --CJK Unified Ideographs Extension B; cjkunifiedideographsextensionb
        or (c >= 0x2A700 and c <= 0x2B73F) --CJK Unified Ideographs Extension C; cjkunifiedideographsextensionc
        or (c >= 0x2B740 and c <= 0x2B81F) --CJK Unified Ideographs Extension D; cjkunifiedideographsextensiond
        or (c >= 0x2B820 and c <= 0x2CEAF) --CJK Unified Ideographs Extension E; cjkunifiedideographsextensione
        or (c >= 0x2CEB0 and c <= 0x2EBEF) --CJK Unified Ideographs Extension F; cjkunifiedideographsextensionf
        or (c >= 0x2F800 and c <= 0x2FA1F) --CJK Compatibility Ideographs Supplement; cjkcompatibilityideographssupplement

end

-- 汉字环境中的(非汉字)合法符号
local legalsymbolinchinese = {
    [0x00b7] = true, -- ·   MIDDLE DOT
    [0x002D] = true, -- -   Hyphen-Minus. Will there be any side effects?
    [0x002F] = true, -- /   Solidus
}

local function islegalsymbolinchinese(c)
    return legalsymbolinchinese[c]
end

function Thirddata.processmystuff(head)
    local n = head
    while n do
        if n.id == glyph_id then
            if  ischinesechar(n.char) then -- 本字是汉字
                local n_prev = n.prev
                if n_prev and n_prev.id == glyph_id then 
                    if ischinesechar(n_prev.char) then -- 前字是汉字
                    elseif not islegalsymbolinchinese(n_prev.char) then --前字不是汉字也不是合法中文语境符号
                        local glue = node_new("glue")
                        glue.width = tex_sp("0.25em")
                        glue.stretch = tex_sp("0.25em")
                        --print("insert space before:", utf8.char(n.char))
                        head, glue = node_insertbefore(head, n, glue)
                    end
                end

                local n_next = n.next
                if n_next 
                and n_next.id == glyph_id 
                and not ischinesechar(n_next.char) 
                and not islegalsymbolinchinese(n_next.char) 
                then
                    local glue = node_new("glue")
                    glue.width = tex_sp("0.25em")
                    glue.stretch = tex_sp("0.25em")
                    --print("insert space after:", utf8.char(n.char))
                    head, glue = node_insertafter(head, n, glue)
                    n = glue.next
                end
            elseif n.char == 0x2014 then -- — — 双符破折号中间加入一个
                local n_prev = n.prev
             if n_prev and n_prev.id == glyph_id and n_prev.char == 0x2014 then 
                    local k_l = node_new("kern")
                    k_l.kern = -n.width / 2
                    head, k_l = node_insertbefore(head, n, k_l)
                    local glyph_n = node_new("glyph")
                    glyph_n.font = font.current()
                    glyph_n.subtype = 0
                    glyph_n.char = 0x2014
                    head, glyph_n = node_insertbefore(head, n, glyph_n)
                    local k_r = node_new("kern")
                    k_r.kern = -n.width / 2
                    head, k_r = node_insertbefore(head, n, k_r)
             end
            end
        end
        n = n.next
    end
    return head, done
end

nodes.tasks.appendaction("processors", "after", "Thirddata.processmystuff")

\stopluacode

% 链接断行规则
\enabledirectives[hyphenators.urls.packslashes] % 避免连续斜杠分到不同行
\sethyphenatedurlnormal{:=?&}
\sethyphenatedurlbefore{?&} 
\sethyphenatedurlafter{:=}
\sethyphenatedurlbefore{/} % 设置在斜杠前断行
% % 设置断行时显示的符号
% \def\hyphenatedurlseparator{↩}

% % 参考文献
% \usebtxdataset[bib.bib]
% \usebtxdefinitions[gbt]
% % \setupbtxlist[aps][distance=.5em,width=auto]



\stopenvironment